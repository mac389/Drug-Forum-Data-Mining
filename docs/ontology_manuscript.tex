% Options for packages loaded elsewhere
\PassOptionsToPackage{unicode}{hyperref}
\PassOptionsToPackage{hyphens}{url}
%
\documentclass[
]{article}
\usepackage{lmodern}
\usepackage{amssymb,amsmath}
\usepackage{ifxetex,ifluatex}
\ifnum 0\ifxetex 1\fi\ifluatex 1\fi=0 % if pdftex
  \usepackage[T1]{fontenc}
  \usepackage[utf8]{inputenc}
  \usepackage{textcomp} % provide euro and other symbols
\else % if luatex or xetex
  \usepackage{unicode-math}
  \defaultfontfeatures{Scale=MatchLowercase}
  \defaultfontfeatures[\rmfamily]{Ligatures=TeX,Scale=1}
\fi
% Use upquote if available, for straight quotes in verbatim environments
\IfFileExists{upquote.sty}{\usepackage{upquote}}{}
\IfFileExists{microtype.sty}{% use microtype if available
  \usepackage[]{microtype}
  \UseMicrotypeSet[protrusion]{basicmath} % disable protrusion for tt fonts
}{}
\makeatletter
\@ifundefined{KOMAClassName}{% if non-KOMA class
  \IfFileExists{parskip.sty}{%
    \usepackage{parskip}
  }{% else
    \setlength{\parindent}{0pt}
    \setlength{\parskip}{6pt plus 2pt minus 1pt}}
}{% if KOMA class
  \KOMAoptions{parskip=half}}
\makeatother
\usepackage{xcolor}
\IfFileExists{xurl.sty}{\usepackage{xurl}}{} % add URL line breaks if available
\IfFileExists{bookmark.sty}{\usepackage{bookmark}}{\usepackage{hyperref}}
\hypersetup{
  hidelinks,
  pdfcreator={LaTeX via pandoc}}
\urlstyle{same} % disable monospaced font for URLs
\usepackage{color}
\usepackage{fancyvrb}
\newcommand{\VerbBar}{|}
\newcommand{\VERB}{\Verb[commandchars=\\\{\}]}
\DefineVerbatimEnvironment{Highlighting}{Verbatim}{commandchars=\\\{\}}
% Add ',fontsize=\small' for more characters per line
\newenvironment{Shaded}{}{}
\newcommand{\AlertTok}[1]{\textcolor[rgb]{1.00,0.00,0.00}{\textbf{#1}}}
\newcommand{\AnnotationTok}[1]{\textcolor[rgb]{0.38,0.63,0.69}{\textbf{\textit{#1}}}}
\newcommand{\AttributeTok}[1]{\textcolor[rgb]{0.49,0.56,0.16}{#1}}
\newcommand{\BaseNTok}[1]{\textcolor[rgb]{0.25,0.63,0.44}{#1}}
\newcommand{\BuiltInTok}[1]{#1}
\newcommand{\CharTok}[1]{\textcolor[rgb]{0.25,0.44,0.63}{#1}}
\newcommand{\CommentTok}[1]{\textcolor[rgb]{0.38,0.63,0.69}{\textit{#1}}}
\newcommand{\CommentVarTok}[1]{\textcolor[rgb]{0.38,0.63,0.69}{\textbf{\textit{#1}}}}
\newcommand{\ConstantTok}[1]{\textcolor[rgb]{0.53,0.00,0.00}{#1}}
\newcommand{\ControlFlowTok}[1]{\textcolor[rgb]{0.00,0.44,0.13}{\textbf{#1}}}
\newcommand{\DataTypeTok}[1]{\textcolor[rgb]{0.56,0.13,0.00}{#1}}
\newcommand{\DecValTok}[1]{\textcolor[rgb]{0.25,0.63,0.44}{#1}}
\newcommand{\DocumentationTok}[1]{\textcolor[rgb]{0.73,0.13,0.13}{\textit{#1}}}
\newcommand{\ErrorTok}[1]{\textcolor[rgb]{1.00,0.00,0.00}{\textbf{#1}}}
\newcommand{\ExtensionTok}[1]{#1}
\newcommand{\FloatTok}[1]{\textcolor[rgb]{0.25,0.63,0.44}{#1}}
\newcommand{\FunctionTok}[1]{\textcolor[rgb]{0.02,0.16,0.49}{#1}}
\newcommand{\ImportTok}[1]{#1}
\newcommand{\InformationTok}[1]{\textcolor[rgb]{0.38,0.63,0.69}{\textbf{\textit{#1}}}}
\newcommand{\KeywordTok}[1]{\textcolor[rgb]{0.00,0.44,0.13}{\textbf{#1}}}
\newcommand{\NormalTok}[1]{#1}
\newcommand{\OperatorTok}[1]{\textcolor[rgb]{0.40,0.40,0.40}{#1}}
\newcommand{\OtherTok}[1]{\textcolor[rgb]{0.00,0.44,0.13}{#1}}
\newcommand{\PreprocessorTok}[1]{\textcolor[rgb]{0.74,0.48,0.00}{#1}}
\newcommand{\RegionMarkerTok}[1]{#1}
\newcommand{\SpecialCharTok}[1]{\textcolor[rgb]{0.25,0.44,0.63}{#1}}
\newcommand{\SpecialStringTok}[1]{\textcolor[rgb]{0.73,0.40,0.53}{#1}}
\newcommand{\StringTok}[1]{\textcolor[rgb]{0.25,0.44,0.63}{#1}}
\newcommand{\VariableTok}[1]{\textcolor[rgb]{0.10,0.09,0.49}{#1}}
\newcommand{\VerbatimStringTok}[1]{\textcolor[rgb]{0.25,0.44,0.63}{#1}}
\newcommand{\WarningTok}[1]{\textcolor[rgb]{0.38,0.63,0.69}{\textbf{\textit{#1}}}}
\usepackage{longtable,booktabs}
% Correct order of tables after \paragraph or \subparagraph
\usepackage{etoolbox}
\makeatletter
\patchcmd\longtable{\par}{\if@noskipsec\mbox{}\fi\par}{}{}
\makeatother
% Allow footnotes in longtable head/foot
\IfFileExists{footnotehyper.sty}{\usepackage{footnotehyper}}{\usepackage{footnote}}
\makesavenoteenv{longtable}
\setlength{\emergencystretch}{3em} % prevent overfull lines
\providecommand{\tightlist}{%
  \setlength{\itemsep}{0pt}\setlength{\parskip}{0pt}}
\setcounter{secnumdepth}{-\maxdimen} % remove section numbering

\date{}

\begin{document}

\hypertarget{header-n0}{%
\subsubsection{\texorpdfstring{Introduction
}{Introduction }}\label{header-n0}}

\hypertarget{header-n3}{%
\subsubsection{Methods}\label{header-n3}}

We analyze the contents of Lycaeum from X to Y. Lycaeum is an online
forum devoted to providing information about substances to encourage
safe and informed use. Please see {[}candyflipping article{]} for a
detailed description of the acquisition of text from Lycaeum.

Calculation of conditional probabilities

\hypertarget{header-n7}{%
\paragraph{Creation of Ontology.}\label{header-n7}}

We created an ontology of novel psychoactive substances for two
purposes, to provide a tool to acceleratee the processing of
unstructured text that describes the use of novel psychoactive substance
and to provide a systematic interface to the data in that manuscript and
developing a system to coregister these data with those contained in
databases about molecular function. These goals correspond to two
different ontologies, one that describes the pattern of appearance of
tokens and one that links those tokens to the physical object to which
they refer. This corresponds to the \emph{use-mention} distinction
{[}Arp reference{]}. We termed the ontologies \emph{ops-use} and
\emph{ops-mention}.

We, following {[}Arp reference{]} created an ontology by first
identifying all nouns from all posts using part-of-speech tagging.
Author MC then manually successively extracted from this list any token
that appeared to describe effects and then any token that appeared to
describe a substance.

We excluded from the ontology strings that were uninterpretable. See
Supplement X for the list of excluded strings.

\hypertarget{header-n12}{%
\paragraph{Creation of Markov Logic Statements.}\label{header-n12}}

A Markov Logic statement has the general form, \$p\textbackslash;
f\textbackslash left(x,y,\textbackslash ldots\textbackslash right)\$,
where \$p \textbackslash in
\textbackslash left(0,1\textbackslash right)\$ represents the chance
that the statement is true and
\$f\textbackslash left(x,y,\textbackslash ldots\textbackslash right)\$
is a decidable statement about entities \$x,y,\textbackslash ldots\$.
The Markov Logic statment \$0.4 \textbackslash;
\textbackslash textrm\{hasToxidrome\}\textbackslash left(X,\textbackslash textrm\{cholinergic\}\textbackslash right)\$
represents that an individual X has a cholinergic toxidrome in 40\% of
possible worlds. We derived the logical statements from the ontology and
probabilities from the frequency with which the terms used in the
ontology appeared in theh Lycaeum corpus. In the preceding example,
\emph{0.4} would represent the fraction of times the cholinergic
toxidrome was mentioned in the Lycaeum corpus.

\begin{Shaded}
\begin{Highlighting}[]
\NormalTok{p ingested(}\DataTypeTok{X}\KeywordTok{,}\NormalTok{s) }\CommentTok{\% probability p that entity X recently substance s}
\NormalTok{p effect(}\DataTypeTok{X}\KeywordTok{,}\NormalTok{e) }\CommentTok{\% probability p that entity X is experiencing effect e}
\end{Highlighting}
\end{Shaded}

Code block X states, in a variant of Prolog called Problog, the three
Markov lgoic statements that represent the probabilistic relationships
between all significantly co-occurring substance-effect,
substance-substance, and effect-effect pairs in our Lycaeum corpus.

\begin{Shaded}
\begin{Highlighting}[]
\NormalTok{ p1 substance\_effect(}\DataTypeTok{X}\KeywordTok{,}\NormalTok{s1}\KeywordTok{,}\NormalTok{e)}\KeywordTok{.}
\NormalTok{ p2 effect\_substance(}\DataTypeTok{X}\KeywordTok{,}\NormalTok{e}\KeywordTok{,}\NormalTok{s1)}\KeywordTok{.}
\NormalTok{ p3 substance\_substance(}\DataTypeTok{X}\KeywordTok{,}\NormalTok{s2}\KeywordTok{,}\NormalTok{s1)}\KeywordTok{.}
\NormalTok{ p4 effect\_effect(}\DataTypeTok{X}\KeywordTok{,}\NormalTok{s2}\KeywordTok{,}\NormalTok{s1)}\KeywordTok{.}
\end{Highlighting}
\end{Shaded}

The first goal states that if individual \$X\$ ingested substance
\$s\emph{1\$, then \$X\$ has a likelihood \$p}1\$ of experiencing effect
\$e\$. We calculated \$p\emph{1\$ as the conditional probability,
\$p\textbackslash left(e\textbar s}1\textbackslash right)\$, by counting
all posts that mentioned effect \$e\$ and substance \$s\emph{1\$ and
dividing that quantity by the count of all posts mentioning substance
\$e\$. We instantiated this goal only for substance-effect pairs whose
conditional probability was statistically significantly greater than
chance, after using the Benjamini-Hochberg correction to adjust the
false disccovery rate to \$0.05\$ . We calculated the second goal, which
is the inverse of the first, and probability,
\$p\textbackslash left(s}1\textbackslash right\textbar e)\$
analogouosly. The order of arguments is important. In general, for any
different occurrences \$x\$ or \$y\$,
\$p\textbackslash left(x\textbar y\textbackslash right)
\textbackslash neq p\textbackslash left(y \textbar{}
x\textbackslash right)\$.

The third goal states that if individual \emph{\$X\$} ingested substance
\$s\emph{1\$, then \$X\$ has a likelihood \$p\$ of having also ingested
substance \$s}2\$. We calculated
\$p\textbackslash left(s\emph{2\textbar s}1\textbackslash right)\$ by
counting all posts that mentioned both substances and then dividing that
quantity by the count of all posts that mentioned \$s\_1\$. The fourth
goal is the third goal but stated for effects.

\hypertarget{header-n23}{%
\paragraph{\texorpdfstring{Determination of statistical significance.
}{Determination of statistical significance. }}\label{header-n23}}

We calculated an empiric probability distribution for the conditional
probabilities in each category (substance-substance, substance-effect or
effect-substance, or effect-effect). For each pair we computed the
\$p\$-value from the pair's rank in the corresponding distribution. We
adjusted the false discovery rate to 0.05 using the Benjamini-Yuetieli
correction. We included in the knowledge base only pairs whose p-value
indicated that the conditional probability

\hypertarget{header-n27}{%
\subsubsection{\texorpdfstring{Results. }{Results. }}\label{header-n27}}

\hypertarget{header-n28}{%
\paragraph{\texorpdfstring{Ontology. }{Ontology. }}\label{header-n28}}

We identified \$33\$ unique effects (Supplemental Table 1) and \$637\$
unique substances \\
(Supplemental Table 2). This yielded
\$1056=33\textbackslash cdot\textbackslash left(33-1\textbackslash right)\$
effect-effect pairs, \$405,132 =
637\textbackslash cdot\textbackslash left(637-1\textbackslash right)\$
substance-substance-pairs, and \$21,021 = 637 \textbackslash cdot 33\$
substance-effect pairs.

\hypertarget{header-n30}{%
\paragraph{Markov Logic Statements.}\label{header-n30}}

\hypertarget{header-n32}{%
\paragraph{\texorpdfstring{Validation of Construction
}{Validation of Construction }}\label{header-n32}}

\hypertarget{header-n33}{%
\paragraph{Synthetic Data Set}\label{header-n33}}

To assess the internal consistency of our large-scale calculations of
conditional probability and

\hypertarget{header-n37}{%
\paragraph{Evaluation of Knowledge Base.}\label{header-n37}}

\hypertarget{header-n38}{%
\paragraph{\texorpdfstring{ }{ }}\label{header-n38}}

\hypertarget{header-n40}{%
\subsubsection{Conclusions}\label{header-n40}}

\begin{center}\rule{0.5\linewidth}{0.5pt}\end{center}

\hypertarget{header-n43}{%
\section{\texorpdfstring{Supplementary Material
}{Supplementary Material }}\label{header-n43}}

\begin{longtable}[]{@{}lllll@{}}
\toprule
emotion & property of toxin & & &\tabularnewline
\midrule
\endhead
global nervous system & somatic sensation & & &\tabularnewline
immunologic & neurologic disorder & & &\tabularnewline
motor & perception & & &\tabularnewline
misclassified & endocrine & & &\tabularnewline
gastrointestinal & sense of self & & &\tabularnewline
song references & use of substance & & &\tabularnewline
cardiovascular & sexual activity & & &\tabularnewline
genitourinary & uninterpretable & & &\tabularnewline
oncologic & respiratory & & &\tabularnewline
written media references & activity references & & &\tabularnewline
chemical process & renal & & &\tabularnewline
metabolic & esp & & &\tabularnewline
hematologic & infectious & & &\tabularnewline
musculoskeletal & cognitive ability & & &\tabularnewline
place references & behavior & & &\tabularnewline
& integumentary & & &\tabularnewline
\bottomrule
\end{longtable}

\end{document}
